% commit do original em 31/mar/15 fa913239e883dfa18e0aa6f44fb0ba6f2db65fd5
\chapter{Funcionalidades do compilador}
\label{cr-features}
\state{NoContent}

\section{Metadados de compilação}
\label{cr-metadata}

Partindo do Haxe 3.0, você pode conseguir a lista dos metadados definidos pelo compilador executando \expr{haxe --help-metas}

\begin{center}
\begin{tabular}{| l | l | l |}
	\hline
	\multicolumn{3}{|c|}{Global metatags} \\ \hline
	Metatag &  Descrição  &  Plataforma \\ \hline
	@:abstract &  Determina a classe subjacente de implementação  como \tref{tipo abstrato}{types-abstract}  &  cs  java \\
	@:access \_(Target path)\_  &   Força o acesso privado ao tipo ou campo do pacote, ver \tref{controle de acesso}{lf-access-control}  &  todas \\
	@:allow \_(Target path)\_  &  Permite acesso privado do tipo ou campo do pacote, ver \tref{controle de acesso}{lf-access-control}  &  todas \\
	@:annotation  &  Definições de anotações (\expr{@interface}) no imports de \expr{-java-lib} serão anotadas com esse metadado. Não tem efeito sobre tipos compilados pelo Haxe &  java \\
    @:arrayAccess  &  Permite \tref{accesso como array}{types-abstract-array-access} em um (tipo) abstrato  &  todas \\
	@:autoBuild \_(Build macro call)\_  &   Estende o metadado \expr{@:build} para todas as classes implementadoras ou exetensoras, ver  \tref{autobuild macro}{macro-auto-build}  &  todas \\
	@:bind  &  Sobrescreve a declaração de classe Swf  &  flash \\
	@:bitmap \_(Bitmap file path)\_  &  \_Embarca os bitmaps dados na classe (deve estender \expr{flash.display.BitmapData})   &  flash \\
	@:build \_(Build macro call)\_  &   Monta uma classe ou enum a partir de uma macro, ver \tref{montagem de tipos}{macro-type-building}  &  todas \\
	@:buildXml  &    &  cpp \\
	@:classCode  &  Usada para injetar código nativo a plataforma em uma classe &  cs  java \\
	@:commutative  &  Declara um operador de abstrato como comutativo  &  todas \\
	@:compilerGenerated  &  Marca um campo como gerado pelo compilador. Não deveria ser usado pelo usuário final &  cs  java \\
	@:coreApi &  Identifica essa classe como uma classe do cerne da API (força a verificação da API)  &  todas \\
    @:coreType  &  Identica um tipo abstrato como {pertencente ao cerne (\tref{core type}{types-abstract-core-type}) de forma que não exija implementação  &  todas \\
	@:cppFileCode  &     &  cpp \\
	@:cppNamespaceCode  &    &  cpp \\
    @:dce  &  Força a \tref{eliminção de código morto (DCE)}{cr-dce} mesmo quando \expr{-dce full} não é especificado  &  todas \\
	@:debug  &  Força a geração de informação de debug a ser gerada no Swf mesmo sem \expr{-debug}   &  flash \\
	@:decl   &     &  cpp \\
	@:defParam  &    &  all \\
	@:delegate  &  Automaticamente inserida por \expr{-net-lib} nos delegados   &  cs \\
	@:depend  &     &  cpp \\
	@:deprecated   &  Automaticamente inserida por  \expr{-java-lib} nos campos de classe anotados com a anotação \expr{@Deprecated}. Não tem efeito em timpos compilados pelo Haxe  &  java \\
	@:event  &  Automaticamente inseridos por \expr{-net-lib} sobre eventos. Não tem efeitos sobre tipos compilados pelo Haxe   &  cs \\
	@:enum  &  Define conjuntos finitos de valores nas definições de abstratos, ver \tref{abstratos de enums}{types-abstract-enum}  &  todas \\
	@:expose \_(?Name=Class path)\_  &  Torna a classe disponível no objeto \expr{window}  ou em \expr{exports} para node.js, ver \tref{exposing Haxe classes for Javascript}{target-javascript-expose} &  js \\
	@:extern  &  Marca o campo como um extern para que não seja gerado  &  all \\
	@:fakeEnum \_(Type name)\_  &  Trata o enum como uma  coleção de valores do tipo especificado  &  all \\
	@:file(File path)  &  Inclui um dado arquivo binário no target Swf e o associa com a classe  (deve estender \expr{flash.utils.ByteArray})  &  flash \\
	@:final  &  Previne uma class classe de ser estendida  &  todas \\
    @:font \_(TTF path Range String)\_  &  Embarca uma fonte TrueType dada na classe (deve estender \expr{flash.text.Font})  &  flash \\
	@:forward \_(lista of nome de campo)\_  &  \tref{Encaminha o acesso de campo}{types-abstract-forward} ao typo subjacente  &  todas \\
    @:from   &  Specifica que o campo do abstrato é uma operação cast (moldagem) a partir do tipo identificado na função, ver \tref{casts implícitos}{types-abstract-implicit-casts}  &  todas \\
	@:functionCode  &     &  cpp \\
	@:functionTailCode  &    &  cpp \\
	@:generic &  Marca uma classe ou campo de classe como \tref{generic}{type-system-generic} para que cada combinação de parâmetro de tipo gere seu próprio tipo/campo  &  todas \\
	@:genericBuild  & Monta instâncias de um tipo usando a macro especificada   &  todas \\
	@:getter \_(nome do campo de classe)\_  &  Gera uma função 'getter' nativa para o campo dado &  flash \\
	@:hack   &  Permite estender classes marcadas como \expr{@:final}  &  todas \\
	@:headerClassCode  &    &  cpp \\
	@:headerCode   &     &  cpp \\
	@:headerNamespaceCode  &    &  cpp \\
	@:hxGen  &  Anota que uma classe extern foi gerada pelo Haxe  &  cs  java \\
    @:ifFeature \_(funcionalidade (feature))\_  &  Faz com que um campo seja mantido pelo \tref{DCE}{cr-dce} se a funcionalidade é dada como parte da compilação  &  todas \\
	@:include &     &  cpp \\
	@:initPackage  &    &  all \\
	@:internal  &  Gera o campo/classe anotada com acesso \expr{internal}  &  cs  java \\
	@:isVar  &  Força um campo físico a ser gerado para propriedades que de outra forma não o exigiriam  &  todas \\
	@:keep   &  Faz com que um campo ou tipo seja mantido \tref{DCE}{cr-dce}  &  all \\
	@:keepInit  &  Faz com que uma classe seja mantida pela \tref{DCE}{cr-dce} mesmo que todos os seus campos sejam removidos  &  todas \\
	@:keepSub &  Estende o metadado \expr{@:keep} para todas as classes implementadoras ou estensoras  &  todas \\
	@:macro  &  \_(deprecated)\_  &  todas \\
	@:meta   &  Internamente usada para marcar um campo de classe como sendo o campo metadado &  todas \\
	@:multiType \_(parâmetros de tipo relevantes)\_  &  Especifica que um abstrato escolhe seu this-type a partir de suas funções \expr{@:to}  &  todas \\
	@:native \_(caminho do tipo de saída)\_  &  Reescreve o caminho de uma classe ou enum durante a geração  &  todas \\
	@:nativeGen  &  Annotates that a type should be treated as if it were an extern definition - platform native  &  cs  java \\
	@:noCompletion  &  Previne o compilador de sugerir o \tref{completamento}{cr-completion} sobre esse campo  &  todas \\
	@:noDebug &  Não gera informação de debug para o Swf mesmo se \expr{-debug} está habilitado   &  flash \\
	@:noDoc  &  Previne um tipo de ser incluído na geração automática de documentação &  todas \\
	@:noImportGlobal  &  Previne um campo estático de ser importado como \expr{import Class.*}  &  todas \\
	@:noPackageRestrict  &  Permite a um módulo ser acessado de todos os targets se localizado em seu primeiro tipo  &  todas \\
	@:noStack &     &  cpp \\
	@:noUsing &  Previne um campo de ser usado com \expr{using}  &  todas \\
	@:notNull &  Declara que um tipo abstrato não aceita \tref{valores \expr{null}}{types-nullability}  &  all \\
	@:ns  &  Internamente usado pelo gerador de Swf para manusear namespaces   &  flash \\
    @:op \_(operação)\_  &   Declara um campo abstrato como um \tref{sobrescritor de operador}{types-abstract-operator-overloading}  &  todas \\
	@:optional  &  Marca o campo de uma estrutura como opcional, ver \tref{argumentos opcionais}{types-nullability-optional-arguments}  &  todas \\
	@:overload \_(especification de função)\_  &  Permite que o campo seja chamado com diferentes tipos de argumento. A especificação de função não pode ser uma expressão  &  todas \\
	@:privateAccess  &  Permite o acesso  privado a tudo para a expressãp anotada  &  todas \\
	@:property  &  Marca um campo de propriedade para ser compilado como uma propriedade nativa de C\# property   &  cs \\
	@:protected  &  Marca um campo de classe como sendo protegido  &  todas \\
	@:public  &  Marca um campo de classe como sendo público  &  todas \\
	@:publicFields  &  Força todos os campos de classe das classes herdeiras a serem públicos & todas \\
	@:readOnly  &  Gera um campo com a palavra-chave \expr{readonly} nativa &  cs \\
	@:remove  &  Faz com que uma interface seja removida de todas as classes implementadoras antes da geração  &  todas \\
    @:require \_(flag de compilação)\_  &  Permite acesso a um campo apenas se a {flag de compilação}{lf-condition-compilation} especificada está habilitada  &  all \\
	@:rtti   &  Insere informações de tipo de tempo de execução, ver \tref{RTTI}{cr-rtti}  &  all \\
	@:runtime  &    &  todas \\
	@:runtimeValue  &  Marks an abstract as being a runtime value  &  all \\
	@:setter \_(nome do campo de classe)\_  &  Gera uma função setter nativa para o dado campo &  flash \\
	@:sound \_(caminho do arquivo)\_  &  Inclui um arquivo \_.wav\_ or \_.mp3\_ dado ao target Swf e o associa com a classe (deve estender \expr{flash.media.Sound})  &  flash \\
	@:struct  &  Marca uma definição de classe como um struct  &  cs \\
	@:suppressWarnings  &  Insere uma anotação SuppressWarnings para a classe Java gerada  &  java \\
	@:throws \_(Type as String)\_  &  Insere uma declaração \expr{throws} à função gerada  &  java \\
    @:to  &  Especifica que o campo do tipo abstrato (abstract) é uma operação cast (fusão) para o tipo identificado na função, ver \tref{casts implícitos}{types-abstract-implicit-casts} & todas \\
    @:transient  &  Adiciona o sinalizador (flag) \expr{transient} ao campo de classe  &  java \\
	@:unbound  &  Interno ao compilador para denotar variável global unbounded &  all \\
	@:unifyMinDynamic  &  Permite a uma coleção de tipos que se unifique com Dynamic  &  todas \\
	@:unreflective  &    &  cpp \\
	@:unsafe  &  Declara uma classe pu um método com a flag \expr{unsafe} de C\#   &  cs \\
	@:usage  &    &  todas \\
	@:volatile  &    &  cs  java \\
\end{tabular}
\end{center}

\section{Eliminação de código morto}
\label{cr-dce}

A eliminação de código morto ou \emph{DCE} (abreviando Dead Code Elimination) é uma funcionalidade que remove código não utilizado da saída. Depois de tipar, o compilador avalia os pontos de entrada (geralmente o método main) e recursivamente determina que tipos e campos são usados. Os campos utilizados são marcados adequadamente e os campos não marcados são removidos de suas classes.

A DCE tem três modos que são definidos quando chamados pela linha de comando:

\begin{description}
    \item[dce std:] Apenas as classes na Biblioteca padrão são afetadas pelo DCE. Isso é o modo padrão para todos os targets.
    \item[-dce no:] Nenhuma eliminação de código morto é executada.
    \item[dce full:] Todas as classes são afetadas pelo DCE.
\end{description}
O algoritmo de DCE funciona bem com código tipado, mas pode falhar quando se usa  \tref{dynamic}{types-dynamic} e\tref{reflection}{std-reflection}. Isso pode exigir a marcação explícita de campos ou classes como sendo utilizadas, ao atribuir lhes os seguintes metadados:

\begin{description}
    \item[\expr{@:keep}:] Se for usado em uma classe, a classe e todos os seus campos não serão afetados pelo DCE.
    \item[\expr{@:keepSub}:] Se usado em uma classe, funciona com \exor{@:keep} na classe anotada e em todas as suas subclasses.
    \item[\expr{@:keepInit}:] Usualmente, uma classe que teve tosos os seus campos removidos pelo DCE (ou é vazia mesmo) é removida do resultado. Ao usar esse metadado, classes vazias são mantidas.
\end{description}

Se uma classe precisa ser marcada com \expr{@:keep} da linha de comando ao invés da edição de seu código fonte, há uma macro de compilação disponível para fazer isso: \expr{--macro keep('type dot path')}. Veja a href{http://api.haxe.org/haxe/macro/Compiler.html#keep}{APIhaxe.macro.Compiler.keep} para detalhes dessa macro. Ela irá marcar o pacote, módulo ou subtipo a ser mantido pelo DCE e incluí-los na compilação.

O compilador automaticamente define um sinalizador (flag) de compilação \expr{dce} com valor entre \expr{ "std"}, \expr{"no"} ou \expr{"full"} dependendo do modo ativado. Isso pode ser usado epara a \tref{compilação condicional}{lf-condition-compilation}.

\trivia{Revisão do DCE}{O DCE foi originalmente implementado no Haxe 2.07. Essa implementação foi considerada uma função a ser usada quando o código era explicitamente tipado. O problema com ela era que diversas funcionalidades, as mais importantes sendo as interfaces, faziam com que todos os campos de classe a serem tipados de forma a verificar a segurança-de-tipos. Isso subverteu o DCE efetivamente todo, demandando sua revisão para o Haxe 2.10}

\trivia{DCE e try.haxe.org}{O DCE para \type{Javascript} viu vastas melhorias quando o website \url{http://try.haxe.org} foi divulgado. A recepção inicial do código gerado em \target{Javascript} foi variada, levando a uma seleção mais delicada de qual código deveria ser eliminado}

\section{Completamento de texto}
\label{cr-completion}

Por conta do grande número de funcionalidades do Haxe e sua poderosa inferência de tipos, \href{http://haxe.org/documentation/introduction/editors-and-ides.html}{IDEs e editores de código} não podem lidar facilmente com o completamento de texto analisando sintaticamente os arquivos de Haxe. O Haxe oferece apoio ao completamento de expressões para IDEs construídos diretamente dentro do compilador graças ao parâmetro de linha de comando \ic{--display}

Vamos dar uma olhada no seguinte exemplo:
\begin{lstlisting}
class Test {
    public static function main() {
        trace("Hello".|
    }
}
\end{lstlisting}

A barra vertical indica a posição do cursor depois que o ponto foi premido. Neste momento, é o trabalho da IDE chamar o compilador do Haxe com os parâmetros usuais de compilação, além de um adicional \ic{--display Test.hx@73}.

Esse parâmetro determina o arquivo no qual nós queremos algum completamento de expressões e a posição em bytes (não em caractetes) do cursor no arquivo. No exemplo dado, você deve obter 73 se você contar caracteres com o CRLF de fins de linha do Windows 

O Haxe desempenhará toda a análise sintática e tipagem como faria para a compilação normal, exceto que não gerará qualquer código. Se ele chega no byte 73 no arquivo \ic{Test.hx} durante a compilação, ele expõe a informação sobre o tipo da expressão à esquerda do ponto.

No nosso caso, isso seria \type{String}. A exceção sendo que no caso de um objeto anônimo ou uma instância de classe, o compilador irá, ao invés, listar os diferentes campos em um XML e imprimí-los para uma possível saída. 


\begin{lstlisting}
<list>
<i n="length"><t>Int</t><d>...</d></i>
<i n="charAt"><t>index : Int -> String</t><d>...</d></i>
<i n="charCodeAt"><t>index : Int -> Int</t><d>...</d></i>
<i n="indexOf">
  <t>value : String -> ?startIndex : Int -> Int</t>
  <d>...</d>
</i>
...
</list>
\end{lstlisting}

Nesse caso, todos os métodos e campos públicos de \type{String} são listados. Essa informação pode ser usada diretamente pela IDE para fornecer dicas de código e completamento de texto.

\paragraph{Completamento de funções}
Esse mecanismo de completamento funciona tanto com ponto quanto com parenteses de abertura, para que você possa conseguir informações de tipos sobre listas de campos e tipos de argumento de chamada de funções:

\begin{lstlisting}
class Test {
    public static function main() {
        trace("Hello".split(|
    }
}
\end{lstlisting}

Isso lhe dara o seguinte resultado: \expr{<type>delimiter : String -> Array<String></type>} \_(por favor, observe que o conteúdo do tipo não contém caracteres de escape de html-escaped na saída verdadeira)\_

\paragraph{Completamento de pacotes}
Ele também funciona para pacotes, escaneando os arquivos .hx disponíveis no caminho-de-classe:

\begin{lstlisting}
import haxe.|
Retornará :

<list>
    <i n="BaseCode"><t></t><d></d></i>
    <i n="EnumFlags"><t></t><d></d></i>
    ...
    <i n="io"><t></t><d></d></i>
    <i n="macro"><t></t><d></d></i>
    <i n="remoting"><t></t><d></d></i>
    ...
</list>
\end{lstlisting}

\paragraph{Manuseio de erros}
Quando executado em modo de completamento de texto, o compilador não exibirá erros, mas ao invés disso, tentará ignorá-los ou se recuperar a partir deles.

Se um erro sem recuperação ocorre enquanto se obtém a completagem de texto, o compilador de Haxe imprimirá a mensagem de erro ao invés da saída de completagem. Você pode, então, tratar qualquer saída não XML como uma mensagem de erro que impede a completagem.

\paragraph{Servidor de cache de compilação}
\since{2.09}

Para conseguir a melhor velocidade tanto para compilação quanto para completagem, você pode usar o parâmetro de linha de comando \ic{--wait} para iniciar um servidor de compilação Haxe; Você também pode usar \ic{-v} para fazer o servidor imprimir um log. Aqui está um exemplo:

\begin{lstlisting}
haxe -v --wait 6000
\end{lstlisting}

Você pode, então, concectar ao servidor Haxe, enviar parâmetros de linha de comando seguidos por um byte 0 e, então, ler a resposta (sejam resultados de completagem de texto ou erros)

Use o parâmetro de linha de comando \ic{--connect} para fazer com que o Haxe envie seus comandos de compilação para o servidor ao invés de executá-los diretamente:

\begin{lstlisting}
haxe --connect 6000 myproject.hxml
\end{lstlisting}

Por favor, observe que você pode usar o parâmetro \ic{--cwd} na primeira linha de comando enviada para mudar o diretório atual de trabalho do servidor de Haxe. Usualmente caminhos de classe e outros arquivos são relativos ao seu projeto.

\paragraph{Como ele funciona}
O servidor de compilação fará cache das seguintes coisas:

\begin{description}
    \item[arquivos analisados sintaticamente (``parseados'')] os arquivos só serão ``parseados'' novamente se eles forem modificados ou se havia um erro de análise
    \item[chamadas a haxelib] os resultados anteriores das chamadas a haxelib serão reutilizados(apenas para completagem: eles serão ignorados quando se faz uma compilação
	\item[módulos tipados] módilos de compilação serão armazenados depois de uma compilação com sucesso e podem ser reutilizados depois em compilações/completagem se nenhuma de suas dependências tiverem sido modificadas
\end{description}

Voê pode conseguir leituras precisas dos tempos gastos pelo compilador e como o servidor de compilação os afetam inserindo \ic{--times} na linha de comando. 

\paragraph{Protocolo}
Como o próximo exemplo de Haxe/Neko mostra, você pode simplesmente conectar na porta do servidor e mandar todos os comandos (um por linha) terminando com um caractere binário 0. Você pode, então, ler os resultados

Macros de outros comandos podem fazer logs de eventos que não são erros. Da linha de comando, nós podemos ver a diferença entre o que é impresso em \ic{stdout} e o que é impresso em \ic{stderr}. Esse não é o caso no modo de socket. De forma a diferenciar entre os dois, mensagens de log (não erros) são prefixados com um caracter \ic{\\x01} e todos os caracteres de newline na mensagem são repostos pelo mesmo caractere \ic{\\x01}.

Warnings e outras mensagens também podem ser considerados erros, mas não erros fatais. Se um erro fatal ocorreu, uma única linha de mensagem \ic{\\x02} será enviada.

Aqui está algum código que se conectará ao servidor e lidará com detalhes do protocolo:

\begin{lstlisting}
class Test {
    static function main() {
		var newline = "\textbackslash\ n";
        var s = new neko.net.Socket();
        s.connect(new neko.net.Host("127.0.0.1"),6000);
        s.write("--cwd /my/project" + newline);
        s.write("myproject.hxml" + newline);
        s.write("\textbackslash\ 000");
		
        var hasError = false;
        for (line in s.read().split(newline))
		{
            switch (line.charCodeAt(0)) {
				case 0x01: 
					neko.Lib.print(line.substr(1).split("\textbackslash\ x01").join(newline));
				case 0x02: 
					hasError = true;
				default: 
					neko.io.File.stderr().writeString(line + newline);
            }
		}
        if (hasError) neko.Sys.exit(1);
    }
}
\end{lstlisting}

\paragraph{Efeito em macros}
O servidor de compilação pode ter alguns efeitos colaterais na \tref{execução de macros}{macro}.

\subsection{Acesso a campos}
\label{cr-completion-field-access}

\subsection{Chamada de argumentos}
\label{cr-completion-call-arguments}

\subsection{Uso}
\label{cr-completion-usage}

\subsection{Posição}
\label{cr-completion-position}

\subsection{Nível superior}
\label{cr-completion-toplevel}

\section{Recursos}
\label{cr-resources}
\flag{fold}{true}

O Haxe oferece um sistema de embarcação de recursos simples que pode ser usado par embarcar arquivos diretamente na aplicação compilada.

Enquanto pode não ser otimizado embarcar coisas grandes como imagens ou música no arquivo da aplicação, isso vem a ser conveniente para embarcar pequenos recursos como configurações ou dados XML.

\subsection{Embarcação(embedding) de recursos}
\label{cr-resources-embed}

Arquivos externos podem ser embarcados usando o argumento de compilação \emph{-resource}

\todo{what to use for listing of non-haxe code like hxml?}
\begin{lstlisting}
-resource hello_message.txt@welcome
\end{lstlisting}

O string depois do símbolo \emph{@} é o \emph{identificador do recurso} que é usado no código para acolher o recurso. Se for omitido (bem como o \emph{@}) o nome do arquivo se tornará o identificador do recurso.

\subsection{Acolhimento de recursos de texto}
\label{cr-resources-getString}

Para acolher o conteúdo de um recurso embarcado, utilizamos o método estático \emph{getString} de \type{haxe.Resource} e passando lhe um \emph{identificador de recursos}:

\haxe{assets/ResourceGetString.hx}

O código acima ira exibir o conteúdo do arquivo \emph{hello_message.txt} que incluímos anteriormente usando wellcome como identificador

\subsection{Acolhimento de recursos binários}
\label{cr-resources-getBytes}

Ainda que não seja recomendado embarcar grandes arquivos binários na aplicação, ainda pode ser útil embarcar dados binários. A representação binária de um recurso embarcado pode ser acessada com o método estático \emph{getBytes} de \type{Haxe.Resource}:

\haxe{assets/ResourceGetBytes.hx}

O tipo do retorno do método \emph{getBytes} é \type{haxe.io.Bytes}, que é um objeto que fornece acesso a bytes individuais dos dados.

\subsection{Detalhes de implementação}
\label{cr-resources-impl}

O Haxe usa a embarcação de recursos nativa da plataforma target se houver uma, se não fornece sua própria implementação.

\begin{itemize}
    \item Recursos em \emph{Flash} são embarcados como definições de ByteArray
    \item Recursos em \emph{C#} são incluídos no assembly compilado.
    \item Recursos em \emph{Java} são empacotados na arquivo JAR resultante
    \item Recursos em \emph{C++} são armazenados nas constantes de array de bites globais.
    \item Recursos em \emph{Javascript} são seriados no formato de serialização do Haxe e armazenados no campo estático da classe \type{haxe.Resource.
    \item Recursos em \emph{Neko} são armazenados como strings em um campo estático da classe \type{haxe.Resource}
\end{itemize}



\section{Informação de tipos em tempo de execução}
\label{cr-rtti}

O compilador do haxe gera informação de tipos em tempo de execução (RTTI - runtime type information) para classes que são marcadas com o metadado \expr{:rtti}  ou para classes que estendam classes com essa marcação. Essa informação é armazenada como um string XML em um campo estático \expr{__rtti} e pode ser processado através da \type{haxe.rtti.XmlParser}. A estrutura resultante é descrita em  \Fullref{cr-rtti-structure}.

\since{3.2.0}

O tipo \type{haxe.rtti.Rtti} foi introduzido de forma a simplificar o trabalho com RTTI. A extração dessa informação é agora muito fácil:

\haxe{assets/RttiUsage.hx}

\subsection{A estrutura da informação de tipos em tempo de execução}
\label{cr-rtti-structure}

\paragraph{Informação geral de tipos}

\begin{description}
    \item[path:] O \tref{caminho de tipo}{define-type-path} para o tipo
    \item[module:] O caminho de tipo do \tref{módulo}{define-module}que contém o tipo
    \item[file:] O caminho com barras (\slash) completo para o arquivo .hx que vonté o tipo. Esse pode ser \expr{null} no caso de não existir o arquivo, e.g., se o tipo é definindo através de uma \tref{macro}{macro}.
    \item[params:] Um array de strings representando os nomes dos \tref{parâmetros de tipo}{type-system-type-parameters} que o tipo tem. A partir do Haxe 3.2.0, isso não inclui as \tref{restrições}{type-system-type-parameter-constraints}.
    \item[ doc:] A documentação do tipo. Essa informação só é disponível se o \tref{sinalizador de compilação}{define-compiler-flag} \expr{-D use_rtti_doc} estiver em ação. De outra forma, ou se o tipo não tiver documentação, o valor é \expr{null}
    \item[isPrivate:] Se o tipo é ou não \tref{privado}{define-private-type}.
    \item[platforms:] Uma lista de strings representando os targets onde o tipo está disponível
     \item[meta:] Os metadados com que o tipo foi marcado.
\end{description}
  
\paragraph{Informaçao de tipo para classes}
\label{cr-rtti-class-type-information}

\begin{description}
    \item[isExtern:] Se a classe é ou não \tref{externalizada}{lf-externs}.
    \item[isInterface:] Se a classe é ou não é uma \tref{interface}{types-interfaces}.
    \item[superClass:] A classe pai da classe defininda por seu caminho de tipo e lista de parâmetros de tipo.
    \item[interfaces:] A lista de interfaces definidas por seu caminho de tipo e lista de parâmetros de tipo
    \item[fields:] A lista de seus \tref{campos de classe }{class-field} membros descritas em\Fullref{cr-rtti-class-field-information}.
    \item[statics:] A lista de campos de classe estáticos, descrita em \Fullref{cr-rtti-class-field-information}.
    \item[tdynamic:] O tipo que está \tref{implementado dinamicamente}{types-dynamic-implemented} pela classe, ou \expr{null} se o tipo não existe.
\end{description}

\paragraph{Informação de tipo enum}

begin{description}
    \item[isExtern:] Se o enum é ou não \tref{extern}{lf-externs}.
    \item[constructors:] A lista dos constructors do enum
\end{description}

\paragraph{Informações para tipo abstratos}

\begin{description}
    \item[to:] um array contendo \tref{casts implicitos}{types-abstract-implicit-casts} para qual o tipo se converte.
    \item[from:] um array contendo \tref{casts implícitos}{types-abstract-implicit-casts} a partir do qual o tipo se converte. 
    \item[impl:] A \tref{informação de tipo de classe}{cr-rtti-class-type-information} da classe de implementação.
    \item[athis:] O \tref{tipo subjacente}{define-underlying-type} do abstrato.
\end{description}

	
\paragraph{Informações para campos de clase}
\label{cr-rtti-class-field-information}

\begin{description}
    \item[name:] O nome do campo.
    \item[type:] o tipo do campo
    \item[isPublic:] Se o campo é ou não \tref{público}{class-field-visibility}.
    \item[isOverride:] Se o campo \tref{sobrescreve}{class-field-override} ou não outro campo.
    \item[doc:] A documentação do campo. Essa informação só é disponível se o  \tref{sinalizador de compilação}{define-compiler-flag} \expr{-D use_rtti_doc} estiver em ação. De outra forma, ou se o tipo não tiver documentação, o valor é \expr{null}.
    \item[get:] O \tref{comportamento de acesso de leitura}{define-read-access} do campo.
    \item[set:] O \tref{comportamento de acesso à escrita}{define-write-access} no campo.
    \item[params:] um array de strings representando os nomes dos \tref{parâmetros de tipo}{type-system-type-parameters} que o campo tem . A partir do Haxe 3.2.0, isso não inclui as \tref{restrições }{type-system-type-parameter-constraints}.
    \item[platforms:] Uma lista de strings representando os targets onde o campo está disponível.
    \item[meta:] Os metadados com os quais o campo foi marcado.
    \item[line:] O número da linha onde o campo é definido. Essa informação só está disponível se o campo tem uma expressão. De outra forma é \expr{null}.
    \item[overloads:] Uma lista dos overloads disponíveis para os campos, ou \expr{null} se não existe overload.
\end{description}

\paragraph{Informações para constructors de enum}
\label{cr-rtti-enum-constructor-information}

\begin{description}
    \item[name:] O nome do constructor.
    \item[args:] a lista dos argumentos que o constructor tem, ou \expr{null} se não há argumentos disponíveis.
    \item[doc:] A documentação do constructor. Essa informação só está disponível se o \tref{sinalizador de compilação}{define-compiler-flag} \expr{-D use_rtti_doc} estiver em ação. De outra forma, ou se o tipo não tiver documentação, o valor é \expr{null}
    \item[platforms:] Uma lista de strings representando os targets onde o campo está disponível.
    \item[meta:] Os metadados com os quais o constructor foi marcado.
\end{description}
