%comit do arquivo original (24/FEV): bcd19df0b0b190b77d52bb03fd923d9c4a263883
\chapter{Detalhes de Targets}
\label{target-details}
\state{NoContent}

\section{Javascript}
\label{target-javascript}
\state{NoContent}

\subsection{Começando com Javascript}
\label{target-javascript-getting-started}

O Haxe pode ser uma ferramenta poderosa para o desenvolvimento de aplicações Javascript. Vamos olhar para o nosso caso.
Este é um exemplo muito simples mostrando o toolchain (cadeia de ferramentas)

Cria uma nova pasta e salva essa classe como \ic{Main.hx}.

\begin{lstlisting}
import js.Lib;
import js.Browser;
class Main {
    static function main() {
        var button = Browser.document.createButtonElement();
        button.textContent = "Click me!";
        button.onclick = function(event) {
            Lib.alert("Haxe é demais");
        }
        Browser.document.body.appendChild(button);
    }
}
\end{lstlisting}

Para compilar, ou execute isso pela linha de comando:

\begin{lstlisting}
haxe -js main-javascript.js -main Main -D js-flatten -dce full
\end{lstlisting}

.. ou crie e execute (com duplo clique) um arquivo chamado \ic{compile.hxml}. Nesse exemplo o arquivo .hxml deve estar no mesmo diretório que a a classe do exemplo.

\begin{lstlisting}
-js main-javascript.js
-main Main
-D js-flatten
-dce full
\end{lstlisting}

A saída será um main-javacript.js, o qual cria e adiciona um botão clicável ao corpo do documento

\paragraph{Rode o Javascript}

Para exibir o a saída em um browser, crie um documento HTM chamado \ic{index.html} e o abra.'

\begin{lstlisting}
<!DOCTYPE html>
<html>
	<body>
		<script src="main-javascript.js">
	</body>
</html>
\end{lstlisting}

\paragraph{Mais informação}

\begin{itemize}
	\item \href{http://api.haxe.org/js/}{Haxe Javascript API docs}
	\item \href{https://developer.mozilla.org/en-US/docs/Web/JavaScript/Reference}{MDN JavaScript Reference}
\end{itemize}

\subsection{Usando as bibliotecas externas do Javascript}
\label{target-javascript-external-libraries}

A Biblioteca Padrão do Haxe vem com muitas classes externas (externs) para o target Javascript. Eles oferecem acessos para as APIs nativas de uma maneira com tipagem segura.
O \tref{mecanismo de externs}{lf-externs} assume que os tipos definidos existem em tempo de execução mas assume nada sobre como e quando esses tipos são definidos.
Para esclarecer, na maior parte dos casos nós temos que adicionar um \ic{<script>}-tag que liga (linka) a biblioteca externa manualmente para o documento HTML.

Um exemplo de uma classe externa é a \href{https://github.com/HaxeFoundation/haxe/blob/development/std/js/JQuery.hx#L83}{jQuery class} da Biblioteca Padrão do Haxe.
Para ilustrar, aqui está uma parte da classe externa:

\begin{lstlisting}
@:initPackage
extern class JQuery implements ArrayAccess<Element> {
	function addClass( className : String ) : JQuery;
	function removeClass( ?className : String ) : JQuery;
	function hasClass( className : String ) : Bool;
	
	@:overload(function(html:String):JQuery{})
	@:overload(function(html:JQuery):JQuery{})
	function html() : String;
..
\end{lstlisting}

Observe que funções podem ser sobrecarregadas (overloaded) para aceitar diferentes tipos de argumentos e retornar valores, usando a metatag \expr{@:overload}. A sobrecarga de funções funciona unicamente em classes externas.
Usando essa externa, podemos usar jQuery assim:

\begin{lstlisting}
new JQuery("#my-div").addClass("brand-success").html("haxe é demais!");
\end{lstlisting}

Além de externs, \tref{Typedefs}{type-system-typedef} podem ser outra ótima maneira de nomear (ou criar um alias para) estruturas complexas de Javascript ou JSON.

O nome do pacotes e classe da classe extern deve ser o mesmo que o definido na biblioteca externa. Se esse não for o caso, reescreva o caminho de uma classe usando \expr{@:native}.

\begin{lstlisting}
package my.application.media;

@:native('external.library.media.video')
extern class Video {
..
\end{lstlisting}

Em Haxe, é possível chamar uma função exposta graças a palavra-chave \expr{untyped}. Isso
pode ser útil em alguns casos, se nós não queremos escrever 'externs'. Qualquer coisa não tipada que seja sintaxe é será gerada como está

\begin{lstlisting}
untyped window.trackEvent("page1");  
\end{lstlisting}

Usando a função mágica \expr{__js__} nós podemos injetar fragmentos de código Javascript puro à saída. Esse código não pode ser validado, então se aceita código inválido na saída, o que é a favor de erros.
Isso poderia, por exemplo, escrever um comentário de Javascript na saída.

\begin{lstlisting}
untyped __js__('// haxe é demais!');
\end{lstlisting}

As opções de compilação padrão também provêem mais fontes de Haxe para ser adicionadas ao projeto:

\begin{itemize}
	\item Para adicionar uma  \tref{Bibliote de Haxe}{haxelib}, use \expr{-lib <library-name>}. Há muitos externs para bibliotecas nativas  populares.
	\item Para adicionar outro caminho de classe, use \expr{-cp <diretório>}.
	\item Para forçar todo um pacote a ser incluído no projeto, use \expr{--macro include('meupackage')} o que incluirá todas as classes declaradas em um dado pacote e subpacote.
\end{itemize}

\subsection{Javascript target Metatags}
\label{target-javascript-metatags}

Essa é a lista de metatags específicas de Javascript. Para mais informação, veja também a lista completa de todas \tref{Metatags internas do Haxe}{cr-metadata}.

\begin{center}
\begin{tabular}{| l | l | l |}
	\hline
	\multicolumn{3}{|c|}{Javascript metatags} \\ \hline
	Metatag &  Descrição & Target \\ \hline
	@:expose \_(?Name=Class path)\_  &  Faz a classe disponível no objeto \expr{window} ou \expr{exports} para o node.js  & js  \\
\end{tabular}
\end{center}

\subsection{Expondo classe do  Haxe para Javascript}
\label{target-javascript-expose}

É possível tornar classes do Haxe ou campos estáticos disponíveis para o uso em Javascript
puro.
Para expor, adicione o metadado \expr{@:expose} às classes ou campos estáticos desejados.

Esse exemplo expõe a classe de Haxe \ic{MyClass}.

\haxe{assets/ClassExpose.hx}

Ele gera a seguinte saída de Javascript:

\begin{lstlisting}
(function ($hx_exports) { "use strict";
var MyClass = $hx_exports.MyClass = function(name) {
	this.name = name;
};
MyClass.prototype = {
	foo: function() {
		return "Greetings from " + this.name + "!";
	}
};
})(typeof window != "undefined" ? window : exports);
\end{lstlisting}

Ao passar 'globais' (como \ic{window} ou \ic{exports}) como parâmetros para as nossas funções anônimas no módulo de Javascript, elas ficam disponíveis, o que permite expor o módulo de Haxe gerado.

Em Javascript puro é possível agora criar uma instância de uma classe e chamar suas funções públicas

\begin{lstlisting}
// Javascript code
var instance = new MyClass('Mark');
console.log(instance.foo()); // registra uma mensagem no console
\end{lstlisting}

O caminho do pacote da classe do Haxe será completamente exposto. Para renomear a classe ou definir um pacote diferente para a classe exposta, use \expr{@:expose("my.package.MyExternalClass")}

\paragraph{Exposição superficial (Shallow expose)}

Quando o código gerado pelo Haxe é parte de um projeto maior de Javascript e encapsulado em um envólucro maior não é sempre necessário expor os tipos de Haxe a variáveis globais.
Compilando o projeto usando \ic{-D shallow-expose} permite que os tipos ou campos estáticos estejam disponíveis apenas no escopo do envólucro gerado.

Quando o código é compilado usando
 \ic{-D shallow-expose}, a saída gerada parecerá com isso:

\begin{lstlisting}
var $hx_exports = $hx_exports || {};
(function () { "use strict";
var MyClass = $hx_exports.MyClass = function(name) {
	this.name = name;
};
MyClass.prototype = {
	foo: function() {
		return "Greetings from " + this.name + "!";
	}
};
})();
var MyClass = $hx_exports.MyClass;
\end{lstlisting}

Nesse padrão, uma delcaração de var é usada para expôr o módulo; ela não escreve no objeto 
 \ic{window} or \ic{exports}. 

\subsection{Carregando classes 'extern' usando a função "require"}
\label{target-javascript-require}
\since{3.2.0}

Plataformas modernas de \target{Javascript}, tais como Node.js provêem uma maneira de carregar
objttos de módulos externos usando a função "require". O Haxe suporta a geração automática de 
declarações "require" para classes \expr{extern}.

Essa funcionalidade pode ser habilitada  pela especificação do metadado \expr{@:jsRequire} para a classe 'extern'
Se a nossa classe \expr{extern} representa um módulo inteiro, passamos um argumento único para o metadado \expr{@:jsRequire} especificando o nome do módulo para carregar:

\haxe{assets/JSRequireModule.hx}

Caso que a nossa classe \expr{extern} represente um objeto dentro de um módulo, o segundo argumento específica um objeto a carregar de um módulo: 

\haxe{assets/JSRequireObject.hx}

O segundo argumento é um caminho-com-pontos, de forma que podemos carregar sub-objetos em uma hierarquia.

Se nós precisamos carregar objetos personalizados em tempo de execução, a função \expr{js.Lib.require} pode ser usada. Ela toma um \expr{String} como seu único argumento e retorna um \expr{Dynamic}, portanto é aconselhado que se atribua uma variável estritamente tipada.

\section{Flash}
\label{target-flash}
\state{NoContent}

\subsection{Começando com Flash}
\label{target-flash-getting-started}

O desenvolvimento de aplicações Flash é realmente fácil com Haxe. Vamos olhar para a nossa primeira amostra de código
Esse é um exemplo básico mostrando muito da cadeia de ferramentas.

Cie uma nova pasta e salve essa classe como \ic{Main.hx}.

\begin{lstlisting}
import flash.Lib;
import flash.display.Shape;
class Main {
    static function main() {
        var stage = Lib.current.stage;
        
        // cria um quadrado cinza centralizado 
        var shape = new Shape();
        shape.graphics.beginFill(0x333333);
		shape.graphics.drawRoundRect(0, 0, 100, 100, 10);
		shape.x = (stage.stageWidth - 100) / 2;
		shape.y = (stage.stageHeight - 100) / 2;
		
		stage.addChild(shape);
    }    
}
\end{lstlisting}

Para compilar isso, ou rode o seguinte da linha de comando

\begin{lstlisting}
haxe -swf main-flash.swf -main Main -swf-version 15 -swf-header 960:640:60:f68712
\end{lstlisting}

.. ou crie e rode (duplo clique) um arquivo chamado \ic{compile.hxml}. Nesse exemplo o arquivo hxml deveria estar no mesmo diretório que a classe de exemplo.

\begin{lstlisting}
-swf main-flash.swf
-main Main
-swf-version 15
-swf-header 960:640:60:f68712
\end{lstlisting}

A saída sera um main-flash.swf com tamanho 960x640 pixels e 60 FPS com cor de fundo laranja e um quadrado cinza no centro.

\paragraph{Exibindo o Flash}

Rode o SWF isolado usando o \href{https://www.adobe.com/support/flashplayer/downloads.html}{Standalone Debugger FlashPlayer}. 

Para exibir o resultado em um browser usando o plugin Flash, crie um documento HTML chamado \ic{index.html} e o abra.

\begin{lstlisting}
<!DOCTYPE html>
<html>
	<body>
		<embed src="main-flash.swf" width="960" height="640">
	</body>
</html>
\end{lstlisting}

\paragraph{Mais informação}

\begin{itemize}
	\item \href{http://api.haxe.org/flash/}{Haxe Flash API docs}
	\item \href{http://help.adobe.com/en_US/FlashPlatform/reference/actionscript/3/}{Adobe Livedocs}
\end{itemize}

\subsection{Embutindo recursos (embedding resources)}
\label{target-flash-resources}

Embarcar recursos é diferente no Haxe, comparado a Actionscript 3. Ao invés de usar \ic{\[embed\]} (metatag-AS3) use \tref{metatags específicas do compilador Flash}{target-flash-metatags} like \ic{@:bitmap}, \ic{@:font}, \ic{@:sound} ou \ic{@:file}.

\begin{lstlisting}
import flash.Lib;
import flash.display.BitmapData;
import flash.display.Bitmap;

class Main {
  public static function main() {
    var img = new Bitmap( new MyBitmapData(0, 0) );
    Lib.current.addChild(img);
  }
}

@:bitmap("relative/path/to/myfile.png") 
class MyBitmapData extends BitmapData { }
\end{lstlisting}

\subsection{Usando bibliotecas externas de Flash}
\label{target-flash-external-libraries}

Para embutir bibliotecas externas \ic{.swf} ou \ic{.swc}, use as seguintes \href{http://haxe.org/documentation/introduction/compiler-usage.html}{opções de compilação}:

\begin{description}
	\item[\expr{-swf-lib <file>}] Embute uma biblioteca SWF ao SWF copilado.
	\item[\expr{-swf-lib-extern <file>}] Adiciona a biblioteca SWF para verificação de tipos mas não a inclui no SWF compilado.
\end{description}

As opções de compilação padrão também provêem mais fontes de Haxe a serem adicionadas ao projeto: 
\begin{itemize}
	\item Para adicionar um outro caminho para a classe use \expr{-cp <directory>}.
	\item Para adicionar uma \tref{Biblioteca Haxelib }{haxelib} use \expr{-lib <library-name>}.
	\item Para forçar todo um pacote a ser incluído no projeto, use \expr{--macro include('mypackage')} o que incluirá todas as classes declaradas em um dado pacote e subpacotes. 
\end{itemize}

\subsection{Metatags do target Flash}
\label{target-flash-metatags}

Esta é a lista das metatags específicas de Flash, para uma lista completa veja \tref{Metatags incluídas no Haxe}{cr-metadata}.

\begin{center}
\begin{tabular}{| l | l | l |}
	\hline
	\multicolumn{3}{|c|}{Flash metatags} \\ \hline
	Metatag &  Descrição  &  Target \\ \hline
	@:bind  &  Sobrepõe a declaração de classes  Swf  &  flash \\
	@:bitmap \_(caminho para o arquivo Bitmap)\_  &  \_Embute dados do bitmap indicado à classe (deve estender \expr{flash.display.BitmapData})   &  flash \\
	@:debug  &  Força informações de debug a ser geradas no Swf mesmo sem \expr{-debug}   &  flash \\
	@:file(Caminho para o arquivo)  &  Incluí um dado arquivo binário ao Swf-target e o associa com a classe  (deve estender \expr{flash.utils.ByteArray})  &  flash \\
	@:font \_(TTF path Range String)\_  &  Embute uma dada TrueTypeFont à  classe (deve estender \expr{flash.text.Font})  &  flash \\
	@:getter \_(Nome do campo de classe )\_  &  Gera uma função getter do dado campo   &  flash \\
	@:noDebug &  Não gera informação de debug para o Swf mesmo que \expr{-debug} esteja habilitado   &  flash \\
	@:ns  &  Internalmente usada pelo gerador de Swf para manipular namespaces   &  flash \\
	@:setter \_(Nome do campo de classe)\_  &  Gera uma função 'setter' nativa no dado campo  &  flash \\
	@:sound \_(caminho para o arquivo)\_  &  Inclui um dado arquivo \_.wav\_ ou \_.mp3\_ ao Swf target e o associa com a classe (must estender \expr{flash.media.Sound})  &  flash \\
\end{tabular}
\end{center}

\section{Neko}
\label{target-neko}

\section{PHP}
\label{target-php}

\section{C++}
\label{target-cpp}
\state{NoContent}

\subsection{Usando 'C++ Defines'}
\label{target-cpp-defines}
\begin{itemize}
	\item ANDROID_HOST
	\item ANDROID_NDK_DIR
	\item ANDROID_NDK_ROOT
	\item BINDIR
	\item DEVELOPER_DIR
	\item HXCPP
	\item HXCPP_32
	\item HXCPP_COMPILE_CACHE
	\item HXCPP_COMPILE_THREADS
	\item HXCPP_CONFIG
	\item HXCPP_CYGWIN
	\item HXCPP_DEPENDS_OK
	\item HXCPP_EXIT_ON_ERROR
	\item HXCPP_FORCE_PDB_SERVER
	\item HXCPP_M32
	\item HXCPP_M64
	\item HXCPP_MINGW
	\item HXCPP_MSVC
	\item HXCPP_MSVC_CUSTOM
	\item HXCPP_MSVC_VER
	\item HXCPP_NO_COLOR
	\item HXCPP_NO_COLOUR
	\item HXCPP_VERBOSE
	\item HXCPP_WINXP_COMPAT
	\item IPHONE_VER
	\item LEGACY_MACOSX_SDK
	\item LEGACY_XCODE_LOCATION
	\item MACOSX_VER
	\item MSVC_VER
	\item NDKV
	\item NO_AUTO_MSVC
	\item PLATFORM
	\item QNX_HOST
	\item QNX_TARGET
	\item TOOLCHAIN_VERSION
	\item USE_GCC_FILETYPES
	\item USE_PRECOMPILED_HEADERS
	\item android
	\item apple
	\item blackberry
	\item cygwin
	\item dll_import
	\item dll_import_include
	\item dll_import_link
	\item emcc
	\item emscripten
	\item gph
	\item hardfp
	\item haxe_ver
	\item ios
	\item iphone
	\item iphoneos
	\item iphonesim
	\item linux
	\item linux_host
	\item mac_host
	\item macos
	\item mingw
	\item rpi
	\item simulator
	\item tizen
	\item toolchain
	\item webos
	\item windows
	\item windows_host
	\item winrt
	\item xcompile
\end{itemize}

\subsection{Usando 'Ponteiros C++'}
\label{target-cpp-pointers}

\section{Java}
\label{target-java}

\section{C\#}
\label{target-cs}

\section{Python}
\label{target-python}
\state{NoContent}
\subsection{Carregando classes externas usando a função "require"





